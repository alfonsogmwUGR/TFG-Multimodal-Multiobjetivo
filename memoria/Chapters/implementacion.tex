\chapter{Implementación}\label{ch:impl}

En este capítulo se abordarán los aspectos más técnicos de este trabajo. Se enumerarán las librerías y herramientas empleadas, y se describirá la estructura que sigue el software implementado, el cual ha sido pensado para reutilizar la mayor cantidad de código posible.

Antes de nada, hay que tener en cuenta que no se trata de un proyecto de software convencional, como podría serlo el desarrollo de una aplicación o de un servicio web. Este es un trabajo de experimentación con algoritmos, con un fuerte componente teórico en lo que se refiere a ciencia de datos y algoritmos de optimización. Por ello, no se ha seguido con rigurosidad los principios de ingeniería del software durante el diseño y desarrollo de este trabajo. Aún así se dedicará un espacio para hablar, aunque sea a un nivel un tanto superficial, de la implementación y la estructura del código.

\section{Librerías utilizadas}

Numpy

Scikit-Learn

Parallel (Multiprocessing?)

jMetal

\section{Estructura del software}

% Diagrama de clases


